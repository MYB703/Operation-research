%%%%%%%%%%%%%%%%%%%%%%%%%%%%%%%%%%%%%%%%%%%%%%%%%%%%%%%%%%%%%%%%%%%%%%%%%%%%%%%%
%%%%%%%%%%%%%%%%%%%%%%%%%%%%%%%%%%%%%%%%%%%%%%%%%%%%%%%%%%%%%%%%%%%%%%%%%%%%%%%%
%%% Template for AIMS Rwanda Assignments         %%%              %%%
%%% Author:   AIMS Rwanda tutors                             %%%   ###        %%%
%%% Email: tutors2017-18@aims.ac.rw                               %%%   ###        %%%
%%% Copyright: This template was designed to be used for    %%% #######      %%%
%%% the assignments at AIMS Rwanda during the academic year %%%   ###        %%%
%%% 2017-2018.                                              %%%   #########  %%%
%%% You are free to alter any part of this document for     %%%   ###   ###  %%%
%%% yourself and for distribution.                          %%%   ###   ###  %%%
%%%                                                         %%%              %%%
%%%%%%%%%%%%%%%%%%%%%%%%%%%%%%%%%%%%%%%%%%%%%%%%%%%%%%%%%%%%%%%%%%%%%%%%%%%%%%%%
%%%%%%%%%%%%%%%%%%%%%%%%%%%%%%%%%%%%%%%%%%%%%%%%%%%%%%%%%%%%%%%%%%%%%%%%%%%%%%%%


%%%%%% Ensure that you do not write the questions before each of the solutions because it is not necessary. %%%%%% 

\documentclass[12pt,a4paper]{article}

%%%%%%%%%%%%%%%%%%%%%%%%% packages %%%%%%%%%%%%%%%%%%%%%%%%
\usepackage{amsmath}
\usepackage{amssymb}
\usepackage{amsthm}
\usepackage{amsfonts}
\usepackage{graphicx}
\usepackage[all]{xy}
\usepackage{tikz}
\usepackage{verbatim}
\usepackage[left=2cm,right=2cm,top=3cm,bottom=2.5cm]{geometry}
\usepackage{hyperref}
\usepackage{caption}
\usepackage{subcaption}
\usepackage{psfrag}

%%%%%%%%%%%%%%%%%%%%% students data %%%%%%%%%%%%%%%%%%%%%%%%
\newcommand{\student}{Yves Bonheur Mugiraneza}
\newcommand{\course}{Operation Research}
\newcommand{\assignment}{2}

%%%%%%%%%%%%%%%%%%% using theorem style %%%%%%%%%%%%%%%%%%%%
\newtheorem{thm}{Theorem}
\newtheorem{lem}[thm]{Lemma}
\newtheorem{defn}[thm]{Definition}
\newtheorem{exa}[thm]{Example}
\newtheorem{rem}[thm]{Remark}
\newtheorem{coro}[thm]{Corollary}
\newtheorem{quest}{Question}[section]

%%%%%%%%%%%%%%  Shortcut for usual set of numbers  %%%%%%%%%%%

\newcommand{\N}{\mathbb{N}}
\newcommand{\Z}{\mathbb{Z}}
\newcommand{\Q}{\mathbb{Q}}
\newcommand{\R}{\mathbb{R}}
\newcommand{\C}{\mathbb{C}}
\usepackage{enumitem}
\usepackage{systeme}

%%%%%%%%%%%%%%%%%%%%%%%%%%%%%%%%%%%%%%%%%%%%%%%%%%%%%%%555
\begin{document}
	
	%%%%%%%%%%%%%%%%%%%%%%% title page %%%%%%%%%%%%%%%%%%%%%%%%%%
	\thispagestyle{empty}
	\begin{center}
		\textbf{AFRICAN INSTITUTE FOR MATHEMATICAL SCIENCES \\[0.5cm]
			(AIMS RWANDA, KIGALI)}
		\vspace{1.0cm}
	\end{center}
	
	%%%%%%%%%%%%%%%%%%%%% assignment information %%%%%%%%%%%%%%%%
	\noindent
	\rule{17cm}{0.2cm}\\[0.3cm]
	Name: \student \hfill Assignment Number: \assignment\\[0.1cm]
	Course: \course \hfill Date: \today\\
	\rule{17cm}{0.05cm}
	\vspace{1.0cm}
	
	\section*{Question 1}
	The problem described is formulated as an integer optimization mode below:\\
	\\
    Decision variables:
    Let $A$ denote the units of product A supposed to be produced and $B$ denote the units of product B supposed to be produced. To produce one unit of product A takes 2 hours and B takes 3 hours,mathematically is $2A+3B$. since it is only possible to produce discrete number of products and the profit has to be maximized and the profit for each unit of product A is 200 \$ and for product B 400 \$. We have $$Max~ z = 200 A + 400 B$$
    Let $y_1$ shows usage extra time, $y1=1$ if additional time is used and 0 if otherwise.
    Let $y2=1$ if more than 10 units of product $A $is produced, and 0 if otherwise. \\
    The model representation:\\

    \begin{align*}
        Max ~z &= 200\,A + 400\,B – 1200\,y_1\\
        \text{s.t } 2A + 3 B    &\leq  40 + 8y_1 \quad(\text{ limited time})\\
        y_2 &\geq \frac{1}{A-10} (A – 10) \quad  (\text{more than ten A product})\\
        B &\geq 5y_2 \quad \text{(at least 5 B)}\\
        A &\geq 0, B \geq 0, integer; y_1, y_2 \{0,1\}
    \end{align*}
	\newpage
	\section*{Question 2}
	
	We are given following LP problem :
	\begin{align}
	Max\quad z = 3x_1 – x_2 + x_3\\
	subjected~to~\quad
	x_1 + 2x_2 + x_3 \leq 4\\
	2x_1 – x _2 + x_3 \geq 1\\
	x_1 \geq 0 , x_2 \geq 0,x_3\leq 0
	\end{align}
	Given problem can normally be solved by solving it's dual using simplex method.\\
	The dual problem is \\
	\begin{align*}
	\text{max } G=4v_1+v_2 \\
	\text{subject to } 
	v_1+2v_2\leq 3\\
	2v_1 - v_2\leq -1\\
	v_1 + v_2 \geq 1\\
	v_1\leq 0, v_2\geq 0
	\end{align*}
	let $w_1=-v_1$ and $w_2 =v_2$ then our dual become
	\begin{align*}
	\text{max } G=-4y_1+y_2 \\
	\text{subject to } 
	-y_1+2y_2\leq 3\\
	-2y_1 - y_2\leq -1\\
	-y_1 + y_2 \geq 1\\
	y_1\geq 0, y_2\geq 0
	\end{align*}
	In order to solve this using simplex method we first have to multiply second constraint by -1:
	\begin{align*}
	\text{max } G=-4y_1+y_2 \\
	\text{subject to } 
	-y_1+2y_2\leq 3\\
	2y_1 + y_2\geq 1\\
	-y_1 + y_2 \geq 1\\
	y_1\geq 0, y_2\geq 0
	\end{align*}
	Then we convert it into canonical form\\
	\begin{align*}
	\text{max } G=-4y_1+y_2+0s_1+ 0s_2+0s_3-MA_1-MA_2 \\
	\text{subject to } 
	-y_1+2y_2+s_1 = 3\\
	2y_1 + y_2 -s_2+A_1 = 1\\
	-y_1 + y_2 -s_3 +A_2 = 1\\
	y_1, y_2,s_1,s_3,s_2,A_1,A_2\geq 0
	\end{align*}
	Then we can proceed by solving the problem using simplex method.\\\\
		
		\underline{\textbf{Simplex tableau 1}}\\\\
		The simplex tableau is as follow: \\
		
		\begin{tabular}{cc|ccccccc|cc}
			\hline 
			$C_B$& $y_B $&$ y_1$ & $y_2$ & $s_1$ & $s_2$ & $s_3$ & $A_1$ & $A_2$ & $b_i$ & $b/x_i$ \\ 
			\hline 
		  0  &  $s_1$ & $-1$ & $2$ & $1$ & $0$ & $0$ & $0$ & $0$ &$3$ & $3/2$\\ 
%			\hline 
		  -M &  $A_1$ & $2$ & $\textcircled 1$ & $0$ & $-1$ & $0$ & $1$ & $0$ & $1$& $1\rightarrow$\\ 
%			\hline 
			-M &  $A_2$ & $-1$ & $1$ & $0$ & $0$ & $-1$ & $0$ & $1$ & $1$ & $1$ \\ 
			\hline 
			& $ c_j$ & $-4$ & $1$ & $0$ & $0$ & $0$ & $-M$ & $-M$ &  & \\ 
			\hline
			& $ z_j$ & $-M$ & $-2M$ & $0$ & $M$ & $M$ & $-M$ & $-M$ & $-2M$ & \\ 
			\hline 
			&  $z_j-c_j$& $4-M$ & $-2M-1$ & $0$ & $M$ & $M$ & $0$ & $0$ & $-2M$ &\\ 
			\hline 
			&  &  & $\uparrow$ &  &  &  &  &  &  & \\ 
%			\hline 
		\end{tabular} \\
		From tableau 1 we can find pivot column which is the column with most negative $z_j-c_j$ number (that is the least number since we are maximizing). \\
		Since in the entries of $Z_j-C_j$ there are still negative number we have to do another iteration.
			The pivot row ,$p$, is $( 2,1,0,-1, 0 , 1 , 0,1)$\\
			New simplex tableau rows will be obtained by :\\
			where $p$ is the pivot row, $R_1,R_2,R_3,R_4,R_5$ are the rows of the tableau
			\begin{align*}
			R_1 &-> R_1-2p\\
			R_2 &-> p\\
			R_3 &-> R_3 - p\\
			\end{align*}
		\textbf{\underline{Simplex Tableau 2}}\\
		The simplex tableau is as follow: \\
		
		\begin{tabular}{cc|cccccc|cc}
			\hline 
			$C_B$& $y_B $&$ y_1$ & $y_2$ & $s_1$ & $s_2$ & $s_3$ & $A_2$ & $b_i$ & $b/x_i$ \\ 
			\hline 
			0  &  $s_1$ & $-5$ & $0$ & $1$ & $2$ & $0$ & $0$  &$1$ & $1/2$\\ 
			%			\hline 
			-M &  $x_2$ & $2$ & $1$ & $0$ & $-1$  & $0$ & $0$ & $1$& $-$\\ 
			%			\hline 
			$-M$ &  $A_2$ & $-3$ & $0$ & $0$ & $\textcircled1$ & $-1$ & $1$ &  $0$ & $0$ $\rightarrow$\\ 
			\hline 
			& $ c_j$ & $-4$ & $1$ & $0$ & $0$  & $-M$ & $-M$ &  & \\ 
			\hline
			& $ z_j$ & $3M+2$ & $1$ & $0$ & $-M-1$ & $M$ & $-M$ & $1$ & \\ 
			\hline 
			&  $z_j-c_j$& $3M+6$ & $0$ & $0$  & $-M-1$ & $M$ & $0$ &  &\\ 
			\hline 
			&  &  &  &  & $\uparrow$ &  &  &  &   \\ 
			%			\hline 
		\end{tabular} \\
		From tableau 2 we can find pivot column which is the column with most negative $z_j-c_j$ number (that is the least number since we are maximizing). \\
		Since in the entries of $Z_j-C_j$ there are still negativeThe profit for each unit of product A is 200 $ and for product B 400 $ number we have to do another iteration.
		The pivot row ,$p$, is $( -3,0,0,1, -1 , 1 , 0)$\\
		New simplex tableau rows will be obtained by :\\
		where $p$ is the pivot row, $R_1,R_2,R_3,R_4,R_5$ are the rows of the tableau
		\begin{align*}
		R_1 &-> R_1-2p\\
		R_2 &-> R_2+p\\
		R_3 &-> p\\
		\end{align*}
		\textbf{\underline{Simplex Tableau 3}}\\
		The simplex tableau is as follow: \\
		
		\begin{tabular}{cc|ccccc|cc}
			\hline 
			$C_B$& $y_B $&$ y_1$ & $y_2$ & $s_1$ & $s_2$ & $s_3$ & $b_i$ & $b/x_i$ \\ 
			\hline 
			$0$  &  $s_1$ & $1$ & $0$ & $$1$$ & $0$ & $\textcircled 2$   &$1$ & $1/2 \rightarrow$\\ 
			%			\hline 
			$1$ &  $x_2$ & $-1$ & $1$ & $0$ & $0$  & $-1$  & $1$& $-$\\ 
			%			\hline 
			$0$ &  $s_2$ & $-3$ & $0$ & $0$ & $1$ & $1$ & $0$  & $-$ \\ 
			\hline 
			& $ c_j$ & $-4$ & $1$ & $0$ & $0$  & $-M$ & $-M$   & \\ 
			\hline
			& $ z_j$ & $-1$ & $1$ & $0$ & $0$ & $-1$ & $1$  & \\ 
			\hline 
			&  $z_j-c_j$& $3$ & $0$ & $0$  & $0$ & $-1$  &  &\\ 
			\hline 
			&  &  &  &  &  & $\uparrow$ &  &    \\ 
			%			\hline 
		\end{tabular} \\
		From tableau 3 we can find pivot column which is the column with most negative $z_j-c_j$ number (that is the least number since we are maximizing). \\
		Since in the entries of $Z_j-C_j$ there are still negative number we have to do another iteration.
		The pivot row ,$p$, is $( 1/2,0,1/2,0, 1 , 0 , 1/2)$\\
		New simplex tableau rows will be obtained by :\\
		where $p$ is the pivot row, $R_1,R_2,R_3,R_4,R_5$ are the rows of the tableau
		\begin{align*}
		R_1 &-> p\\
		R_2 &-> R_2+p\\
		R_3 &-> R_3+p\\
		\end{align*}
		\textbf{\underline{Simplex Tableau 3}}\\
		The simplex tableau is as follow: \\
		
		\begin{tabular}{cc|ccccc|cc}
			\hline 
			$C_B$& $y_B $&$ y_1$ & $y_2$ & $s_1$ & $s_2$ & $s_3$ & $b_i$ & $b/x_i$ \\ 
			\hline 
			$0$  &  $s_3$ & $1/2$ & $0$ & $1/2$ & $0$ & $1$ & $1/2$&\\ 
			%			\hline 
			$1$ &  $y_2$ & $-1/2$ & $1$ & $1/2$ & $0$  & $0$  & $3/2$ & $-$\\ 
			%			\hline 
			$0$ &  $As2$ & $-5/2$ & $0$ & $1/2$ & $1$ & $0$ & $1/2$  & $-$ \\ 
			\hline 
			& $ c_j$ & $-4$ & $1$ & $0$ & $0 $ & $-M$ & $-M $  & \\ 
			\hline
			& $ z_j$ & $-1/2 $& $1 $& $0$ & $1/2$ & $0$ & $1.5$ &\\ 
			\hline 
			&  $z_j-c_j$& $7/2$ & $0$ & $1/2$  & $0$ & $0$  &  &\\ 
			\hline 
			&  &  &  &  &  & $\uparrow$ &  &    \\ 
			%			\hline 
		\end{tabular} \\
	
		Since all $z_j-c_j\geq 0$ there is no further iteration. Hence the optimal solution is arrived with value of variable $y_1=0$ and $y_1=3/2$.
	
	\begin{enumerate}[label=(\alph*)]
		\item And we are asked to check if the point $(x_1 , x _2 , x _3 ) = (6/5, 7/5, 0)$ an optimal solution.\\
		To check whether this point is an optimal solution we can use duality theorem.\\
		\begin{thm}
			Duality theorem: If an optimal solution exists to either the primal or
			symmetric dual LP problem, then the other program has also an optimal
			solution and the two objective functions have the same value. Formally, let
			$X= (x_1,\cdots, x_n )^T$ be the optimal solution to the primal and
			$W= ( w_1 . . . , w_n )$ be the optimal solution to the symmetric dual. Then, it follows that 
			\begin{align*}
			C^TX&= W^TB\\
			X&=(6/5,7/5,0)\\
			C^T&=(3,-1,1)\\
			B^T&=(4,1)\\
			\left( \begin{array}{cccc} 3 & -1 & 1 \end{array} \right)\left( \begin{array}{c} 6/5 \\\\ 7/5 \\\\ 0 \end{array} \right)= \left(\begin{array}{cc}4 & 1 \end{array}\right)\left( \begin{array}( 0\\\\ 3/2 \end{array} \right)
			\end{align*}
			
		\end{thm}
		
		\item And we are asked to check if the point $(x_1 , x_2 , x_3 ) = (1/2, 0, 0)$ an optimal solution.
	\end{enumerate} 

    \newpage
	\section*{Question 3}
	We have been asked to study the following LP problem using duality concept.
	\begin{align}
	Max\quad z = c_1 x_1 + c_2 x_2\\
	subjected~to~\quad
	–x_1 + x_2 \leq 1\\
	x_1 + 2x_2 \leq 2\\
	2x_1 + x_2 \geq 0\\
	2x_1 – 2x_2 \leq 1\\
	x_1 \geq 0 , x_2 \geq 0
	\end{align}
	
	\begin{enumerate}[label=(\alph*)]
		\item  determine the primal solution where the dual variables are given by the vector $V=(0, 1/3, 0, 2/3)^T$.\\
		From the theorem of complementary slackness which states that :
		\begin{thm}\label{thm1}
			Assume that $\bar{X}$ is a feasible solution to the primal problem and $\bar{X}$ is a feasible solution to its dual form. Then $\bar{X}$ and
			$\bar{W}$ are optimal solutions to their respective problem if and only if the
			complementary slackness constraints,
			$$\bar{W}^T( B - A\bar{X})= 0 \,\text{or} \,\bar{X}^T (A^T \bar{W}-C )= 0$$
		\end{thm}
		we can slove to get the primal solution that is values of $X$.\\
		\begin{align}
		&\bar{W}^T( B - A\bar{X})= 0\\
		\text{since }W&=(0, 1/3, 0, 2/3)^T\\
		 A\bar{X} &= \systeme{–x_1 + x_2,x_1 + 2x_2,2x_1 + x_2,2x_1-2x_2} ~\text{and}~ B=(1,2,0,1)^T\\
		 \text{then}&\left( \begin{array}{cccc} 0 & \frac{1}{3} & 0 & \frac{2}{3} \\\\ \end{array} \right)\left( \begin{array}{c} 1-(–x_1 + x_2) \\\\ 2-(x_1 + 2x_2) \\\\ 0-(2x_1 + x_2)\\\\ 1-(2x_1-2x_2)\end{array} \right)=0\\
		 &\Rightarrow \systeme{-\frac{x_1}{3}-\frac{2x_2}{3} + \frac{2}{3}=0,-\frac{4x_1}{3}+\frac{4x_2}{3} + \frac{2}{3}=0}\\
		 \end{align}
		 and when we solve this system of equation we get $x_1=1$ and $x_2=\frac{1}{2}$.
		 
		 \item 
	\end{enumerate}

\end{document}
